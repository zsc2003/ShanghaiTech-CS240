\problem{}

Given a set $C$, a collection of subsets of $C$ and an integer $k \geq 1$, the Set-Packing problem asks if there are $k$ subsets from the collection which are pairwise disjoint (i.e. no two sets share an element). Show that the Set-Packing problem is NP-complete.

\solution{}



To prove that the Set-Packing problem is NP-complete, we need to demonstrate two things:

\begin{enumerate}
    \item \textbf{Set-Packing is in NP}: We need to show that a given solution can be verified in polynomial time.
    \item \textbf{Set-Packing is NP-hard}: We need to reduce a known NP-complete problem to Set-Packing in polynomial time.
\end{enumerate}

\subsection*{1. Set-Packing is in NP}
A problem is in NP if a solution can be verified in polynomial time. For the Set-Packing problem, given a collection of subsets and an integer \( k \), a "solution" would be a list of \( k \) subsets. We can verify whether these \( k \) subsets are pairwise disjoint by checking each pair of subsets to ensure they do not share any element. There are \( \binom{k}{2} \) pairs of subsets to check, and for each pair, we can check for intersection in time proportional to the sizes of the subsets involved. Thus, if the subsets and \( k \) are polynomial in the size of the input, this verification process is polynomial, confirming that Set-Packing is in NP.

\subsection*{2. Set-Packing is NP-hard}
To show that Set-Packing is NP-hard, we reduce from a known NP-complete problem. A convenient choice is the Independent Set problem, which is well-known to be NP-complete.

\subsubsection*{Independent Set Problem}
Given a graph \( G = (V, E) \) and an integer \( k \), the problem asks whether there exists a set of \( k \) vertices in \( G \) such that no two vertices in the set are adjacent.

\subsubsection*{Reduction from Independent Set to Set-Packing}
\begin{enumerate}
    \item \textbf{Construction}: Given an instance of the Independent Set problem, i.e., a graph \( G = (V, E) \) and an integer \( k \), construct an instance of Set-Packing as follows:
    \begin{itemize}
        \item For each vertex \( v \) in \( V \), create a subset \( S_v \) containing all the edges that are incident to \( v \) in \( G \). Specifically, \( S_v = \{e \in E \mid v \in e\}\).
        \item Let the collection of subsets be \( \{S_v \mid v \in V\} \).
        \item The goal is to find whether there exists a set of \( k \) subsets from this collection that are pairwise disjoint.
    \end{itemize}
    \item \textbf{Correctness of the Reduction}:
    \begin{itemize}
        \item If there is an independent set of size \( k \) in \( G \), then the corresponding subsets \( \{S_v \mid v \) is in the independent set\(\} \) are pairwise disjoint. This is because if two vertices \( u \) and \( v \) are independent (not adjacent), there are no edges shared between \( S_u \) and \( S_v \).
        \item Conversely, if there exists a set of \( k \) pairwise disjoint subsets in the Set-Packing instance, the vertices corresponding to these subsets form an independent set in \( G \). No two subsets share an edge, which means their corresponding vertices do not share an edge and thus are not adjacent.
    \end{itemize}
    \item \textbf{Polynomial Time}: The reduction constructs a subset for each vertex and assigns the incident edges to each subset, both of which can be accomplished in polynomial time relative to the size of \( V \) and \( E \).
\end{enumerate}

\subsection*{Conclusion}
Given that we can verify solutions in polynomial time and that we can reduce the Independent Set problem (which is NP-complete) to Set-Packing in polynomial time, it follows that Set-Packing is NP-complete. This shows that it is not only a complex problem but one for which no polynomial-time algorithm is likely to exist unless P = NP.



\newpage