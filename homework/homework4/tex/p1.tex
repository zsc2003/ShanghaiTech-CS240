\problem{}

Given a set $C$, a collection of subsets of $C$ and an integer $k \geq 1$, the Set-Packing problem asks if there are $k$ subsets from the collection which are pairwise disjoint (i.e. no two sets share an element). Show that the Set-Packing problem is NP-complete.

\solution{}

1. Set-Packing is in NP.\\
For a collection of subsets and an integer \( k \), we can directly check each pair of subsets to verify whether they are pairwise disjoint. There are \( \binom{k}{2} \) pairs of subsets to check, and for each pair, we can check whether they have the same element in $O(|C|)$ time by iterating through the elements, where $|C|$ is the number of elements in the set $C$. So we can verify the solution is valid or not in $O\left(\binom{k}{2}|C|\right)$ time.\\
So Set-Packing $\in$ NP.\\

2. Independent-Set $\leq_p$ Set-Packing.\\
We have known that the Independent-Set problem is NP-complete.\\
Given a graph \( G = (V, E) \) and an integer \( k \), the Independent Set problem asks whether there exists a set of \( k \) vertices in \( G \) such that no two vertices in the set are adjacent.\\

<1> "$\Rightarrow$":\\
For an instance of Independent-Set problem, construct an instance of Set-Packing as follows:\\
Let each node in the graph be a subset in the Set-Packing problem. For each node \( v \) in \( V \), create a subset \( S_v \) containing all the edges that are incident to \( v \) in \( G \). Specifically, \( S_v = \{e \in E \mid v \in e\} \).\\
Since it is a yes instance of Independent-Set problem, there exists a set of \( k \) vertices in \( G \) such that no two vertices in the set are adjacent. This means that there exists a set of \( k \) subsets in the Set-Packing problem that are pairwise disjoint. So it is a yes instance of Set-Packing problem.\\

<2> "$\Leftarrow$":\\
For an instance of Set-Packing problem, construct an instance of Independent-Set as follows:\\
Let each subset in the Set-Packing problem be a node in the graph. For each subset \( S_v \) in the Set-Packing problem, create a node \( v \) in the graph.\\
Since it is a yes instance of Set-Packing problem, there exists a set of \( k \) subsets in the collection which are pairwise disjoint. This means that there exists a set of \( k \) vertices in \( G \) such that no two vertices in the set are adjacent. So it is a yes instance of Independent-Set problem.\\

<3> Polynomial Time:\\
The reduction constructs a subset for each vertex and assigns the incident edges to each subset, both of which can be accomplished in polynomial time relative to the size of \( V \) and \( E \).\\

So we have proved that Independent Set $\leq_p$ Set-Packing.\\

Since Independent-Set $\in$ NP-complete; Set-Packing $\in$ NP; Independent-Set $\leq_p$ Set-Packing.\\
Therefore, Set-Packing $\in$ NP-complete.\\

So above all, we have proved that Set-Packing is a NP-complete problem.

\newpage