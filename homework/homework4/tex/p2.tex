\problem{}

Given a Boolean CNF (conjunctive normal form) formula $\phi$ and an integer $k \geq 1$, the Stingy-SAT problem asks whether the formula has a satisfying assignment in which at most $k$ variables are set to true. Prove that Stingy-SAT is NP-complete.

\solution{}

1. Stingy-SAT is in NP.\\
We can varify a CNF formula \(\phi\) is satisfied or not in polynomial time.\\
And then check the number of variables set to true is no more than \(k\) in $O(n)$ time, where $n$ is the length of $\phi$.\\
So the total time complexity is polynomial.\\
So Stingy-SAT $\in$ NP.\\

2. SAT $\leq_p$ Stingy-SAT.\\
We can reduce SAT to Stingy-SAT.\\
Given a SAT problem, we additionally check whether the number of variables set to true is no more than \(k\).\\

<1> "$\Rightarrow$":\\
If the SAT problem with firmula $\phi$ with $k$ variables is satisfied, then the number of variables set to true is no more than \(k\).\\
So the Stingy-SAT $(\phi,k)$ is satisfied.\\

<2> "$\Leftarrow$":\\
If the Stingy-SAT $(\phi,k)$ is satisfied, then without the constrains of $k$, the SAT problem is also satisfied.\\

<3> Polynomial Time:\\
The construction only needs to check the number of variables set to true, which is in $O(n)$ time.\\
So it could be done in polynomial time.\\
So we have proved that SAT $\leq_p$ Stingy-SAT.\\

Since SAT $\in$ NP-complete; Stingy-SAT$\in$ NP; SAT $\leq_p$ Stingy-SAT.\\
Therefore, Stingy-SAT $\in$ NP-complete.\\

So above all, we have proved that Stingy-SAT is a NP-complete problem.

\newpage