\problem{}
Given three sequences $L_1, L_2$ and $L$, design an efficient algorithm to check if $L_1$ and $L_2$ can be interleaved to produce $L$.  For example, the sequences $L_1 = \texttt{aabb}$ and $L_2 = \texttt{cba}$ can be interleaved into sequence $L = \texttt{acabbab}$, but $L_1$ and $L_2$ cannot be interleaved into sequence $L = \texttt{aaabbbc}$.  Analyze the time and space complexity of your algorithm as a function of the lengths of $L_1$ and $L_2$.

\solution{}

Let $dp(i,j)$ be a boolean value, which means whether the first $i$ characters of $L_1$ and the first $j$ characters of $L_2$ can be interleaved to produce the first $i+j$ characters of $L$.

The initial condition is $dp(0,0)=1$, as if $L_1$ and $L_2$ are both empty, then they can be interleaved to produce an empty sequence. \\
And set $dp(i,0)=1$ if $L_1[1:i]$ can be interleaved to produce $L[1:i]$, otherwise $dp(i,0)=0$.\\
Similarly, set $dp(0,j)=1$ if $L_2[1:j]$ can be interleaved to produce $L[1:j]$, otherwise $dp(0,j)=0$.\\
And other state are set to be $0$.\\

Then we have the following recursive formula:\\

1. If $L_1[i] == L[i+j]$, then $dp(i,j) \gets dp(i,j) \lor dp(i-1,j)$.\\
As if $L_1[i]$ is the same as $L[i+j]$, then if $L_1[1:i-1]$ and $_2[1:j]$ can be interleaved to produce $L[1:i+j-1]$, then $L_1[1:i]$ and $L2[1:j]$ can be interleaved to produce $L[1:i+j]$.\\

2. If $L_2[j] == L[i+j]$, then $dp(i,j) \gets dp(i,j) \lor dp(i,j-1)$.\\
As if $L_2[j]$ is the same as $L[i+j]$, then if $L_1[1:i]$ and $L2[1:j-1]$ can be interleaved to produce $L[1:i+j-1]$, then $L_1[1:i]$ and $L2[1:j]$ can be interleaved to produce $L[1:i+j]$.\\
The above two conditions can hold at the same time. If they all hold, the two situations should be all considered.\\

Finally, the answer is $dp(|L_1|,|L_2|)$. As if the first $|L_1|$ characters of $L_1$ and the first $|L_2|$ characters of $L_2$ can be interleaved to produce the first $|L_1|+|L_2|$ characters of $L$, then $L_1$ and $L_2$ can be interleaved to produce $L$.\\

The pseudo-code is shown in Algorithm \ref{alg:problem-4}.\\
For loop in the pseudo-code in form `$\textbf{for } i \gets 1 \textbf{ to } n$' represents $i=1,2,\cdots,n$ sequentially.
\begin{algorithm}
    \caption{Interleaved the object sequence}
    \begin{algorithmic}[1]
    \State \textbf{Input:} Two sequences $L_1, L_2$ and $L$.
    \State \textbf{Output:} Whether $L_1$ and $L_2$ can be interleaved to produce $L$.
        
    \State $dp(0,0) \gets 1$
    \For{$i \gets 1$ \textbf{to} $|L_1|$}
        \If{$L_1[i] == L[i]$}
            \State $dp(i,0) \gets 1$
        \EndIf
    \EndFor
    \For{$j \gets 1$ \textbf{to} $|L_2|$}
        \If{$L_2[j] == L[j]$}
            \State $dp(0,j) \gets 1$
        \EndIf
    \EndFor
    \For{$i \gets 1$ \textbf{to} $|L_1|$}
        \For{$j \gets 1$ \textbf{to} $|L_2|$}
            \State $dp(i,j) \gets 0$
            \If{$L_1[i] == L[j]$}
                \State $dp(i,j) \gets dp(i,j) \lor dp(i-1,j)$
            \EndIf
            \If{$L_2[j] == L[j]$}
                \State $dp(i,j) \gets dp(i,j) \lor dp(i,j-1)$
            \EndIf
        \EndFor
    \EndFor
    
    \State \textbf{return} $dp(|L_1|,|L_2|)$
    \end{algorithmic}
    \label{alg:problem-4}
\end{algorithm}\\
The time complexity is $O(nm)$.\\
The space complexity is $O(nm)$.\\
If we use the scrolling array method to optimize, the space complexity can be reduced to $O(n+m)$.\\
Where $n$ is the length of $L_1$, $m$ is the length of $L_2$.\\

\newpage