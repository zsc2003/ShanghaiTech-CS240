\problem{}
Suppose you have $n$ coins, where the $i$'th coin has a size $c_i > 0$, and many piggy banks, each with a uniform capacity \(V\), such that \(V \geq \max(c_i)\).  You want to place all the coins into the minimum number of piggy banks.  To do this you use the following greedy strategy.

Start with one active piggy bank.  Then, sequentially go through the coins, attempting to place each coin into any active piggy bank where it fits. If a coin does not fit into any active piggy bank, take a new piggy bank and make it active.  The algorithm is shown below.  Prove this algorithm is a 2-approximation, i.e. it uses at most two times the minimum number of piggy banks needed for all the coins.

\begin{algorithm}
\caption{Piggy Bank Coin Packing}
\begin{algorithmic}[1]
\label{alg:piggy}
\State \textbf{Input:} Sizes of coins \(c_1, c_2, \ldots, c_n\); size of piggy bank \(V\)
\State \textbf{Output:} Number of piggy banks used

\State Initialize \(b \leftarrow 1\) \Comment{current number of active piggy banks}
\State Initialize \(P_1, P_2, \ldots \leftarrow 0\) \Comment{space used in each piggy bank}
\For{\(i = 1\) to \(n\)}
    \If{\(\exists j \leq b\) such that \(P_j + c_i \leq V\)}
        \State Choose any \(j\) with \(P_j + c_i \leq V\)
        \State \(P_j \leftarrow P_j + c_i\) \Comment{put \(c_i\) in \(j\)-th active piggy bank}
    \Else
        \State \(b \leftarrow b + 1\)
        \State \(P_b \leftarrow c_i\) \Comment{open a new piggy bank and put \(c_i\) in it}
    \EndIf
\EndFor
\end{algorithmic}
\end{algorithm}

\solution{}

Suppose that the result generated by the Algorithm 1 mentioned above: \\
The number of piggy banks used is $k$, and each bank has a size of $P_1,\cdots,P_k$.\\
And suppose that the optimal solution totally uses $k^*$ piggy banks.\\

And we could prove that: $\forall i,j\in \{1,\cdots,k\}, i\neq j$, then we have $P_i+P_j > V$.\\
This could be proved by contradiction:\\
Suppose that there exist $i,j\in \{1,\cdots,k\}, i\neq j$, and $P_i+P_j \leq V$.\\
Then we could get that the total size of the coins in the $i$-th and $j$-th piggy banks is $P_i+P_j$, which is less than or equal to $V$.\\
So we can just merge the $i$-th and $j$-th piggy banks into one piggy bank, and the total number of piggy banks used would be $k-1$, which contradicts the algorithm.\\

So we could use this property to get that:
\begin{equation}
    \sum_{i=1}^{n}c_i=\sum_{i=1}^{k}P_i\geq \dfrac{k}{2}V
    \label{eq:greedy}
\end{equation}

And then consider the optimal solution, we could get that:
\begin{equation}
    \sum_{i=1}^{n}c_i\leq k^*\cdot V
    \label{eq:optimal}
\end{equation}

And combine equation \ref{eq:greedy} and equation \ref{eq:optimal}, we could get that:
$$k^*\cdot V \geq \sum_{i=1}^{n}c_i \geq \dfrac{k}{2}V$$
i.e.
$$\dfrac{k}{2}\leq k^*$$

So above all, we have proved that the algorithm is a 2-approximation.\\

\newpage