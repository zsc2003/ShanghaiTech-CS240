\problem{}
Suppose there are $n$ values in an array, and we want to sort the array using ``flipping" operations.  A flip takes two inputs $i$ and $j$, with $1 \leq i \leq j \leq n$, and reverses the order of the values between indices $i$ and $j$ in the array.  For example, if the array is $[1,1,5,3,4]$, then a flip with indices 2 and 5 changes the array to $[1,4,3,5,1]$.  Assume a flip with inputs $i$ and $j$ has cost $j-i$.  

\hspace*{\fill}

\noindent (a) Assume first that all the values in the array are either 1 or 2.  Design an algorithm which sorts the array using $O(n \log n)$ cost, and analyze its cost.  \\
(Hint: mergesort)

\hspace*{\fill}

\noindent (b) Now suppose the array contains arbitrary values.   Design an algorithm which has $O(n \log^2 n)$ expected cost, and analyze its cost.  Note that your algorithm is allowed to make randomized choices. \\
(Hint: quicksort, and the previous algorithm)

\solution{}

(a) Similarly to the process of merge sort, we firstly suppose that the array split from the middle, and get two ordered arraies, then we make sure the merged array is ordered.\\
After split from the middle, we suppose the two small arraies are ordered. Since the array is combined with only $1$ and $2$, so there must has a line of demarcation to seperate $1$s and $2$s.
And we flip the origin array from the left small array's demarcation line(right side, i.e. the first $2$) to the right small array's demarcation line(left size, i.e. the last $1$).\\
This would take the cost of $O(n)$, where $n$ is the length of the origin array.\\
With this flip, the array must has a continously $1$s followed with continously $2$s, which means that the flipped array is ordered. And we can just recursively repeat this process like merge sort.\\

Then we can get the cost relationship: suppose the cost of sorting an array with length $n$ is $T(n)$, then we have
$$T(n)=2T\left(\dfrac{n}{2}\right)+O(n)$$
According to the Master Theorem, we can get that $T(n)=O(n\log n)$.\\
So above all, we can sort this kind of arraies with cost $O(n\log n)$.

(b) We do this similarly to the quicksort. For each range $[a,b]$, we can select a random number, such as the first element in the range as the pivot.\\
Then we consider the rest of the element, if the element $<$ pivot, set it into $1$, otherwise set it into $2$.\\
Then the range $[a+1,b]$ is conbined by only $1$ and $2$. And we can sort them with the method in (a), and its cost is $O(n\log n)$, where $n$ is the length of the range. Suppose the boarder line of $1$s and $2$s are at index $a+i$(i.e. the number of numbers that are less than the pivot is $i$). 
Then we flip the range $[a,i]$ to make sure all element less than the pivot are at the left of the pivot, and all element greater than the pivot are at the right of the pivot. This will cost $O(n)$, so the total cost during the step is $O(n\log n)$.\\
Then we recursively do this to the left part and the right part.\\
So the cost is that $C(n)=C(i)+C(n-i-1)+O(n\log n)$, where $i$ is the number of numbers that are less than the pivot.\\
Suppose that the expected cost is $T(n)$, then we have
\begin{align*}
    T(n)  &= \sum_{i=0}^{n-1}\dfrac{1}{n}[T(i)+T(n-i-1)+O(n\log n)] \\
    nT(n) &= 2\sum_{i=0}^{n-1}T(i) + O(n^2\log n)            \text{\ \ \ \ \ \ \ \ \ \ \ \ \ \ \ \ \ (equation 1)} \\
    (n-1)T(n-1) &= 2\sum_{i=0}^{n-2}T(i) + O((n-1)^2\log (n-1))      \text{\ \ \ (equation 2)}
\end{align*}
Let equation 1 minus equation 2, we have
\begin{align*}
    nT(n) &= (n+1)T(n-1) + O(n\log n)\\
    \dfrac{T(n)}{n+1} &= \dfrac{T(n-1)}{n} + O(\dfrac{\log n}{n+1})\\
    \dfrac{T(n-1)}{n} &= \dfrac{T(n-2)}{n-1} + O(\dfrac{\log (n-1)}{n}) = \dfrac{T(n-2)}{n-1} + O(\dfrac{\log n}{n})\\
    \cdots\\
    \dfrac{T(2)}{3} &= \dfrac{T(1)}{2} + O(\dfrac{\log 2}{3}) = \dfrac{T(1)}{2} + O(\dfrac{\log n}{3})
\end{align*}
Sum up these equations, we can get that
$$\dfrac{T(n)}{n+1}=O\left[logn\cdot \sum_{i=3}^{n+1}\dfrac{1}{i}\right]$$
From the knowledge of mathematics, we know that
$$\sum_{i=1}^{n}\dfrac{1}{i}=\ln n + \gamma = O(\log n)$$
So we have 
$$\dfrac{T(n)}{n+1}=O(\log^2n)$$
i.e.
$$T(n)=O(n\log^2n)$$

So above all, we can get that the expected cost is $O(n\log^2n)$.

\newpage