\problem{}
Suppose you have two arrays each with $n$ values, and all the values are unique.  You want to find the median of the $2n$ values, i.e. the $n$'th smallest value.  To do this you can make queries to either array, where a query for $k$ to an array returns the $k$'th smallest value in that array.  Give an algorithm which finds the median of the two arrays using $O(\log n)$ queries.  

\solution{}

Suppose $n=2^k$.\\
For the two arrays, $A,B$, we first sort them to be the ascending order.\\
Then to find the median, we can do following steps.
\begin{itemize}
    \item 1. Query the $\frac{|A|}{2}$-th smallest element of $A$, set it to be $a$, and the $\frac{|B|}{2}$-th smallest element of $B$, set it to be $b$.
    \item 2. Split array $A$ into the left half and the right half, and so does array $B$.
    \begin{itemize}
        \item If $a<b$, let the $A$ be the right half of origin $A$, $B$ be the left half.
        \item Else $a>b$, let $A$ be the left half, $B$ be the right half.
    \end{itemize}
\end{itemize}

We can recursively following the step $1$ and step $2$ repeatively in $(k-1)$ times, where $k=\log_{2}{n}$.\\
And the for the $k$'s comparison, it must has $|A|=|B|=2$, we take the second small number among the $4$ numbers, and that number is the $n$-th smallest number of the two arrays.\\

Proof:\\
For the $i$-th step, W.L.O.G, let $a<b$. Then the $n$-th smallest number must be bigger than $a$, but less than $b$, so the $n$-th smallest number must be on the right part of $A$, or the left part of $B$.\\
And the number we take is the $\left[\left(\dfrac{1}{2}\sum\limits_{i=2}^{k}2^i\right)+2=(n-2)+2=n\right]$-th smallest number.\\
And we totally has $O(2(k-1)+1)=O(\log n)$ queries.\\

So above all, with $O(\log n)$ queries, we can find the median of the two arrays.\\

\newpage
